\documentclass[11pt,a4paper]{article}

\usepackage{tadoc}

%\usepackage{amsmath,amssymb,amsfonts}
\usepackage{xcolor}
%\usepackage{graphicx}

\usepackage{minted}
\usemintedstyle{autumn}
\setminted{linenos,breaklines,tabsize=4,xleftmargin=1.5em}
%\usepackage{framed}

%\renewcommand{\multirowsetup}{\centering} 

\title{VE482 --- Introduction to\\ Operating Systems}
\subtitle{Project 1 (Compile guide)}
\author{\href{mailto:liuyh615@sjtu.edu.cn}{Yihao Liu}}
\semester{Fall}
\year{2018}
\blockinfo{
	\begin{center}
		\textbf{Goals of the guide }
	\end{center}
	\begin{itemize}\itemsep .25cm
		\item Install and use LLVM / Clang
		\item Use GNU make / CMake
		\item Submit on JOJ
	\end{itemize}
}

% whether or not to display the instructor line
\noinstructor

%\pagenumbering{gobble}

\begin{document}

\maketitle

\section{Introduction}

You're going to know how to build your project compatible to JOJ and submit it in this guide. \smallskip

\noindent We are using \texttt{llvm/clang} to compile and test your program on JOJ, and we provide two build tools: \texttt{GNU make} and \texttt{CMake}, you can choose either of them in this project.

\section{LLVM / Clang}

\subsection{Introduction}

\texttt{clang} is now widely used as a substitute of \texttt{gcc}, it has GCC compatibility, fast compiles and low memory use, and expressive diagnostics. \smallskip

\noindent So it's a good choice to install and use \texttt{clang} to compile and run your projects locally. In addition, Minix 3 only supports \texttt{clang} in default, instead of \texttt{gcc}. \bigskip

\noindent The LLVM Project is a collection of modular and reusable compiler and toolchain technologies. Despite its name, LLVM has little to do with traditional virtual machines. The name "LLVM" itself is not an acronym; it is the full name of the project. \smallskip

\noindent Find more information on \url{https://llvm.org/}. \bigskip

\noindent The Clang project provides a language front-end and tooling infrastructure for languages in the C language family (C, C++, Objective C/C++, OpenCL, CUDA, and RenderScript) for the LLVM project. \smallskip

\noindent Find more information on \url{https://clang.llvm.org/}.

\subsection{Installation}

On Windows, you can install and use \texttt{clang} for normal c projects, but this project needs some POSIX standard supports, while Windows doesn't have a full implementation of the standard, so you're recommended to switching to Linux. \smallskip

\noindent On most Linux distributions, \texttt{clang} can be found in the package manager. For example, for Debian (Ubuntu / Linux Mint), you can install it with

\begin{minted}{shell}
$ sudo apt install clang
\end{minted}
%$

\noindent On Mac OS X, \texttt{clang} is the default compiler installed. The \texttt{gcc} and \texttt{g++} commands are only alias of it. \smallskip

\noindent On Minix 3, \texttt{clang} is also the default compiler, but you need to install it yourselves.

\begin{minted}{shell}
$ pkgin install binutils
$ pkgin install clang
\end{minted}

\subsection{Sanitizers}

\texttt{clang} provides some sanitizers to detect memory leaks, undefined behaviors and etc.

\subsubsection{AddressSanitizer}

AddressSanitizer is a fast memory error detector. It consists of a compiler instrumentation module and a run-time library. The tool can detect the following types of bugs:

\begin{itemize}
	\item Out-of-bounds accesses to heap, stack and globals
	\item Use-after-free
	\item Use-after-return
	\item Use-after-scope
	\item Double-free, invalid free
	\item Memory leaks
\end{itemize}

\noindent Find more information on \url{https://clang.llvm.org/docs/AddressSanitizer.html}.

\subsubsection{UndefinedBehaviorSanitizer}

UndefinedBehaviorSanitizer (UBSan) is a fast undefined behavior detector. UBSan modifies the program at compile-time to catch various kinds of undefined behavior during program execution, for example:

\begin{itemize}
	\item Using misaligned or null pointer
	\item Signed integer overflow
	\item Conversion to, from, or between floating-point types which would overflow the destination
\end{itemize}

\noindent Find more information on \url{https://clang.llvm.org/docs/UndefinedBehaviorSanitizer.html}.

\section{Build tools}

\subsection{GNU Make}

Here we have a sample \texttt{Makefile} for you. Fill in \texttt{MUMSH\_SRC} and then run \texttt{make} to build your project.

\inputminted{makefile}{Makefile}


\subsection{CMake}

Here we have a sample \texttt{CMakeLists.txt} for you. Add files in \texttt{SOURCE\_FILES} and make sure you have \texttt{cmake} installed on your system (provided on Linux / Mac OS X / Minix 3 by package manager).

\inputminted{cmake}{CMakeLists.txt}

\noindent Then run the following commands to build your project. If you are using \texttt{CLion}, it will be automatically configured with \texttt{CMakeLists.txt}.

\begin{minted}{shell}
$ mkdir cmake-build-debug && cd cmake-build-debug
$ cmake -DCMAKE_C_COMPILER=clang . 
$ make
\end{minted}
%$

\section{Submission on JOJ}

You should archive everything in a tarball (\texttt{*.tar}) and submit on JOJ. Remember to select your build tool first, and your \texttt{Makefile} and \texttt{CMakeLists.txt} must be in the first level directory. 

\end{document} 
